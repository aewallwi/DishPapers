
%%\documentclass[12pt,preprint]{aastex}

%% manuscript produces a one-column, double-spaced document:

%% \documentclass[10pt,manuscript]{aastex}

%% preprint2 produces a double-column, single-spaced document:
\documentclass[preprint2,iop,numberedappendix,twocolappendix,appendixfloats]{emulateapj}
%% \documentclass[preprint2,iop]{aastex}

%% \documentclass[preprint2,longabstract]{aastex}

%% \usepackage{ccaption}
%% \captionstyle{\raggedright}
\usepackage[caption=false]{subfig}
\usepackage{amsmath}
\usepackage{footnote}
\usepackage{url}
\bibpunct{(}{)}{;}{a}{}{,} 
\captionsetup{belowskip=12pt,aboveskip=4pt}
\setlength{\textfloatsep}{10pt plus 1.0pt minus 2.0pt}
\newcommand{\dif}{\mathrm{d}}
%% \renewcommand*{\thefootnote}{\fnsymbol{footnote}}

\def\nar{{New~A~Rev.}}          % New Astronomy Review
\def\pasa{{PASA}}               % Publications of the Astron. Soc. of Australia

%% \bibliographystyle{mn2e}
%% \bibliographystyle{apj}

\shorttitle{Effects of Beam Chromaticity in EoR Power Spectra Measurements}
\shortauthors{Thyagarajan et~al.}

\def\ASU{\altaffilmark{1}}
\def\ASUtxt{\altaffiltext{1}{Arizona State University, School of Earth and Space Exploration, Tempe, AZ 85287, USA}}

\def\myemail{\altaffilmark{*}}
\def\myemailtxt{\altaffiltext{*}{e-mail: t\_nithyanandan@asu.edu}}

\def\UW{\altaffilmark{2}}
\def\UWtxt{\altaffiltext{2}{University of Washington, Department of Physics, Seattle, WA 98195, USA}}

\def\SKASA{\altaffilmark{3}}
\def\SKASAtxt{\altaffiltext{3}{Square Kilometre Array South Africa (SKA SA), Park Road, Pinelands 7405, South Africa}}

\def\RU{\altaffilmark{4}}
\def\RUtxt{\altaffiltext{4}{Department of Physics and Electronics, Rhodes University, Grahamstown 6140, South Africa}}

\def\CfA{\altaffilmark{5}}
\def\CfAtxt{\altaffiltext{5}{Harvard-Smithsonian Center for Astrophysics, Cambridge, MA 02138, USA}}

\def\ANU{\altaffilmark{6}}
\def\ANUtxt{\altaffiltext{6}{Australian National University, Research School of Astronomy and Astrophysics, Canberra, ACT 2611, Australia}}

\def\CAASTRO{\altaffilmark{7}}
\def\CAASTROtxt{\altaffiltext{7}{ARC Centre of Excellence for All-sky Astrophysics (CAASTRO)}}

\def\Haystack{\altaffilmark{8}}
\def\Haystacktxt{\altaffiltext{8}{MIT Haystack Observatory, Westford, MA 01886, USA}}

\def\MIT{\altaffilmark{9}}
\def\MITtxt{\altaffiltext{9}{MIT Kavli Institute for Astrophysics and Space Research, Cambridge, MA 02139, USA}}

\def\Curtin{\altaffilmark{10}}
\def\Curtintxt{\altaffiltext{10}{International Centre for Radio Astronomy Research, Curtin University, Perth, WA 6845, Australia}}

\def\Victoria{\altaffilmark{11}}
\def\Victoriatxt{\altaffiltext{11}{Victoria University of Wellington, School of Chemical \& Physical Sciences, Wellington 6140, New Zealand}}

\def\UWisc{\altaffilmark{12}}
\def\UWisctxt{\altaffiltext{12}{University of Wisconsin--Milwaukee, Department of Physics, Milwaukee, WI 53201, USA}}

\def\UMichigan{\altaffilmark{13}}
\def\UMichigantxt{\altaffiltext{13}{University of Michigan, Department of Atmospheric, Oceanic and Space Sciences, Ann Arbor, MI 48109, USA}}

\def\UMelbourne{\altaffilmark{14}}
\def\UMelbournetxt{\altaffiltext{14}{The University of Melbourne, School of Physics, Parkville, VIC 3010, Australia}}

\def\USydney{\altaffilmark{15}}
\def\USydneytxt{\altaffiltext{15}{The University of Sydney, Sydney Institute for Astronomy, School of Physics, NSW 2006, Australia}}

\def\CASS{\altaffilmark{16}}
\def\CASStxt{\altaffiltext{16}{CSIRO Astronomy and Space Science (CASS), PO Box 76, Epping, NSW 1710, Australia}}

\def\Tata{\altaffilmark{17}}
\def\Tatatxt{\altaffiltext{17}{National Centre for Radio Astrophysics, Tata Institute for Fundamental Research, Pune 411007, India}}

\def\RRI{\altaffilmark{18}}
\def\RRItxt{\altaffiltext{18}{Raman Research Institute, Bangalore 560080, India}}

\def\NRAO{\altaffilmark{19}}
\def\NRAOtxt{\altaffiltext{19}{National Radio Astronomy Observatory, Charlottesville and Greenbank, USA}}

\def\UWA{\altaffilmark{20}}
\def\UWAtxt{\altaffiltext{20}{International Centre for Radio Astronomy Research, University of Western Australia, Crawley, WA 6009, Australia}}

%% \definenote[thanks][conversion=set 2]

\begin{document}

\title{Effect of Beam Chromaticity on Foregrounds in Wide-Field Measurements of Redshifted 21~cm Power Spectra}

%% Use \author, \affil, and the \and command to format
%% author and affiliation information.
%% Note that \email has replaced the old \authoremail command
%% from AASTeX v4.0. You can use \email to mark an email address
%% anywhere in the paper, not just in the front matter.
%% As in the title, use \\ to force line breaks.

%% Author list
\author{
%% Lead Authors
Nithyanandan~Thyagarajan\ASU\myemail,
TBD
% Daniel~C.~Jacobs\ASU,
% Judd~D.~Bowman\ASU,
% N.~Barry\UW,
% A.~P.~Beardsley\UW,
% G.~Bernardi\SKASA$^,$\RU$^,$\CfA,
% F.~Briggs\ANU$^,$\CAASTRO,
% R.~J.~Cappallo\Haystack, 
% P.~Carroll\UW,
% B.~E.~Corey\Haystack, 
% % A.~A.~Deshpande\RRI, 
% A.~de~Oliveira-Costa\MIT,
% Joshua~S.~Dillon\MIT,
% D.~Emrich\Curtin,
% % B.~M.~Gaensler\USydney$^,$\CAASTRO, 
% A.~Ewall-Wice\MIT,
% L.~Feng\MIT,
% R.~Goeke\MIT,
% L.~J.~Greenhill\CfA,
% B.~J.~Hazelton\UW, 
% J.~N.~Hewitt\MIT,
% N.~Hurley-Walker\Curtin,
% M.~Johnston-Hollitt\Victoria,
% D.~L.~Kaplan\UWisc, 
% J.~C.~Kasper\UMichigan$^,$\CfA, 
% Han-Seek Kim\UMelbourne$^,$\CAASTRO,
% P.~Kittiwisit\ASU,
% E.~Kratzenberg\Haystack, 
% E.~Lenc\USydney$^,$\CAASTRO,
% J.~Line\UMelbourne$^,$\CAASTRO,
% A.~Loeb\CfA,
% C.~J.~Lonsdale\Haystack, 
% M.~J.~Lynch\Curtin, 
% B.~McKinley\UMelbourne$^,$\CAASTRO,
% S.~R.~McWhirter\Haystack,
% D.~A.~Mitchell\CASS$^,$\CAASTRO, 
% M.~F.~Morales\UW, 
% E.~Morgan\MIT, 
% A.~R.~Neben\MIT,
% D.~Oberoi\Tata, 
% A.~R.~Offringa\ANU$^,$\CAASTRO, 
% S.~M.~Ord\Curtin$^,$\CAASTRO,
% Sourabh Paul\RRI,
% B.~Pindor\UMelbourne$^,$\CAASTRO,
% J.~C.~Pober\UW,
% T.~Prabu\RRI, 
% P.~Procopio\UMelbourne$^,$\CAASTRO,
% J.~Riding\UMelbourne$^,$\CAASTRO,
% A.~E.~E.~Rogers\Haystack, 
% A.~Roshi\NRAO, 
% N.~Udaya~Shankar\RRI, 
% Shiv~K.~Sethi\RRI,
% K.~S.~Srivani\RRI, 
% R.~Subrahmanyan\RRI$^,$\CAASTRO, 
% I.~S.~Sullivan\UW,
% M.~Tegmark\MIT,
% S.~J.~Tingay\Curtin$^,$\CAASTRO, 
% C.~M.~Trott\Curtin$^,$\CAASTRO,
% M.~Waterson\Curtin$^,$\ANU,
% R.~B.~Wayth\Curtin$^,$\CAASTRO, 
% R.~L.~Webster\UMelbourne$^,$\CAASTRO, 
% A.~R.~Whitney\Haystack, 
% A.~Williams\Curtin, 
% C.~L.~Williams\MIT,
% C.~Wu\UWA,
% J.~S.~B.~Wyithe\UMelbourne$^,$\CAASTRO
}

%Institutional footnotes (typeset, then rearrange here to be in order)
\ASUtxt
% \UWtxt
% \SKASAtxt
% \RUtxt
% \CfAtxt
% \ANUtxt
% \CAASTROtxt
% \Haystacktxt
% \MITtxt
% \Curtintxt
% \Victoriatxt
% \UWisctxt
% \UMichigantxt
% \UMelbournetxt
% \USydneytxt
% \CASStxt
% \Tatatxt
% \RRItxt
% \NRAOtxt
% \UWAtxt
\myemailtxt

%% \clearpage

\begin{abstract}

\end{abstract}
 
\keywords{cosmology: observations --- dark ages, reionization, first stars --- large-scale structure of universe --- methods: statistical --- radio continuum: galaxies --- techniques: interferometric}

\section{Introduction}\label{intro}

The period in the history of the Universe characterized by the transition of neutral hydrogen in the intergalactic medium (IGM) to a fully ionized state due to the formation of radiating objects such as the first stars and galaxies is referred to as the Epoch of Reionization (EoR). This is an important period of nonlinear growth of matter density perturbations and astrophysical evolution leading to the large scale structure observed currently in the Universe. And yet, this period in the Universe's history has remained poorly probed to date with observations. 

The redshifted neutral hydrogen from the IGM in this epoch has been identified to be one of the most promising and direct probes of the EoR \citep{sun72,sco90,mad97,toz00,ili02}. Numerous experiments using low frequency radio telescopes targeting the redshifted 21~cm line from the spin-flip transition of H{\sc i} have become operational such as the Murchison Widefield Array \citep[MWA;][]{lon09,bow13,tin13}, the Precision Array for Probing the Epoch of Reionization \citep[PAPER;][]{par10}, the Low Frequency Array \citep[LOFAR;][]{van13} and the Giant Metrewave Radio Telescope EoR experiment \citep[GMRT;][]{pac13}. These instruments have sufficient sensitivity for a statistical detection of the EoR signal via estimating the spatial power spectrum of the redshifted H{\sc i} temperature fluctuations \citep{bea13,thy13}. These instruments are intended to be precursors and pathfinders to the next generation of low frequency radio observatories such as the Hydrogen Epoch of Reionization Array\footnote{\url{http://reionization.org/}} (HERA; DeBoer~et~al.~2015) and the Square Kilometre Array\footnote{\url{https://www.skatelescope.org/}} (SKA). These next-generation instruments will advance the capability from a mere statistical detection of the signal to a direct three-dimensional tomographic imaging of the H{\sc i} during the EoR. 

The most significant challenge to low frequency EoR observations arises from the extremely bright Galactic and extragalactic foreground synchrotron emission which are $\sim 10^4$ times stronger than the desired EoR signal \citep{dim02,ali08,ber09,ber10,gho12}. All the current and future instruments rely on the inherent differences in spatial isotropy and spectral smoothness between the EoR signal and the foregrounds to extract the EoR power spectrum \citep[see, e.g.,][]{fur04b,mor04,zal04,san05,fur06,mcq06,mor06,wan06,gle08}. 

When expressed in the coordinate system of power spectrum measurements described by the three-dimensional wavenumber ($k$), the foreground emission is restricted to a wedge-shaped region commonly referred to as the {\it foreground wedge} \citep{bow09,liu09,liu14a,liu14b,dat10,liu11,gho12,mor12,par12b,tro12,ved12,dil13,pob13,thy13,dil14} whereas the EoR signal has spherical symmetry due to its isotropy which appears elongated along line of sight $k$ modes due to peculiar velocity effects when dominated by matter density perturbations during early stages of reionization. 

The extreme dynamic range required to subtract foregrounds precisely demands high precision modeling of foregrounds as observed by modern wide-field instruments \citep{thy15a,thy15b}. Their studies of effects of wide-field measurements of EoR power spectra have demonstrated the {\it pitchfork} effect wherein foreshortening of baselines causes a prominent enhancement of foreground power near the horizon limits of the {\it foreground wedge} as well in the contamination beyond. In this paper, we explore yet another phenomenon that inherently extends foreground power beyond the {\it wedge}, namely, that arising from the chromaticity of the antenna power pattern. % We specifically use the current design of 14~m dish of HERA as an example in our study.

This paper is organized as follows. \S\ref{sec:HERA} introduces the HERA instrument. A brief summary of the delay spectrum technique used extensively in this analysis and the recently confirmed {\it pitchfork} effect are presented in \S\ref{sec:delay-spectrum}. \S\ref{sec:sim} describes foreground simulations including antenna beam pattern and all-sky foreground models. \S\ref{sec:beam-chromaticity} investigates the effects of chromaticity of antenna beam on the resulting delay power spectrum and the cosmologically motivated constraints it places on dish reflectometry. Our findings are summarized in \S\ref{sec:summary}.

\section{The Hydrogen Epoch of Reionization Array}\label{sec:HERA}

DeBoer~et~al.(2016)

\section{Delay Spectrum}\label{sec:delay-spectrum}

A brief description of the delay spectrum technique \citep{par12a,par12b} is provided here. We borrow the notation used in \citet{thy15a}. 

{\it Visibilities} measured by an interferometer are given by \citep{van34,zer38,tho01}:
\begin{align}\label{eqn:obsvis}
  V_b(f) &= \iint\limits_\textrm{sky} A(\hat{\boldsymbol{s}},f)\,I(\hat{\boldsymbol{s}},f)\,e^{-i2\pi f\frac{\boldsymbol{b}\cdot\hat{\boldsymbol{s}}}{c}}\,\dif\Omega,
\end{align}
where, $\boldsymbol{b}$ is the vector joining antenna pairs (commonly referred to as the baseline vector), $\hat{\boldsymbol{s}}$ is the unit vector denoting direction on the sky, $f$ denotes frequency, $c$ is the speed of light, $\dif\Omega$ is the solid angle element to which $\hat{\boldsymbol{s}}$ is the unit normal vector, $I(\hat{\boldsymbol{s}},f)$ and $A(\hat{\boldsymbol{s}},f)$ are the sky brightness and antenna's directional power pattern, respectively, as a function of $\hat{\boldsymbol{s}}$ and $f$. The {\it delay spectrum}, $\tilde{V}_b(\tau)$, is defined as the inverse Fourier transform of $V_b(f)$ along the frequency coordinate:
\begin{align}\label{eqn:delay-transform}
  \tilde{V}_b(\tau) &\equiv \int V_b(f)\,W(f)\,e^{i2\pi f\tau}\,\dif f,
\end{align}
where, $W(f)$ is a spectral weighting function which can be chosen to control the quality of the delay spectrum \citep{ved12,thy13}, and $\tau$ represents the signal delay between antenna pairs:
\begin{equation}\label{eqn:delay}
  \tau = \frac{\boldsymbol{b}\cdot\hat{\boldsymbol{s}}}{c}.
\end{equation}

The delay spectrum has a close resemblance to cosmological H~{\sc i} spatial power spectrum. Thus, the delay power spectrum is defined as:
\begin{align}\label{eqn:delay-power-spectrum}
  P_\textrm{d}(\boldsymbol{k}_\perp,k_\parallel) &\equiv |\tilde{V}_b(\tau)|^2\left(\frac{1}{\Omega\Delta B}\right)\left(\frac{D^2\Delta D}{\Delta B}\right)\left(\frac{\lambda^2}{2k_\textrm{B}}\right)^2,
\end{align}
with
\begin{align}
  \boldsymbol{k}_\perp &\equiv \frac{2\pi(\frac{\boldsymbol{b}}{\lambda})}{D}, \label{eqn:kperp-baseline}\\
  k_\parallel &\equiv \frac{2\pi\tau\,f_{21}H_0\,E(z)}{c(1+z)^2}, \label{eqn:kprll-delay}
\end{align}
where, $\Delta B$ is the bandwidth, $\lambda$ is the wavelength of the band center, $k_\textrm{B}$ is the Boltzmann constant, $f_{21}$ is the rest frame frequency of the 21~cm spin flip transition of H~{\sc i}, $z$ is the redshift, $D\equiv D(z)$ is the transverse comoving distance, $\Delta D$ is the comoving depth along the line of sight corresponding to $\Delta B$, and $h$, $H_0$ and $E(z)\equiv [\Omega_\textrm{M}(1+z)^3+\Omega_\textrm{k}(1+z)^2+\Omega_\Lambda]^{1/2}$ are standard terms in cosmology, and following \citet{par14},
\begin{align}
  \Omega\Delta B &= \iint \left|A(\hat{\boldsymbol{s}},f)\,W(f)\right|^2\,\dif\Omega\,\dif f.
\end{align}

In this paper, we use $\Omega_\textrm{M}=0.27$, $\Omega_\Lambda=0.73$, $\Omega_\textrm{K}=1-\Omega_\textrm{M}-\Omega_\Lambda$, $H_0=100\,$km$\,$s$^{-1}\,$Mpc$^{-1}$, and $P_\textrm{d}(\boldsymbol{k}_\perp,k_\parallel)$ is in units of K$^2 (h^{-1}$~Mpc$)^3$.

It was recently discovered that in wide-field measurements diffuse foreground emission from wide off-axis angles appear enhanced in the delay spectrum near the edges of the {\it foreground wedge} even on wide antenna spacings \citep{thy15a}. Called the {\it pitchfork} effect, this arises due to severe foreshortening of baseline vectors towards the horizon along joining the antenna pairs thereby enhancing their sensitivity to large scale structures in these directions. Since delay spectrum maps directions on the sky to delay bins, the emission from large scales near the horizon appears enhanced in delay bins near the horizon limits of the {\it foreground wedge}. Since these delay modes lie adjacent to those considered sensitive to the EoR signal, they cause a significant contamination of line-of-sight modes critical for EoR signal detection. These findings were confirmed in MWA observations \citep{thy15b}.

It was also demonstrated in these studies that design of antenna power pattern, specifically its amplitude near the horizon, is an important tool in mitigating foreground contamination caused by these wide-field effects. A dish characterized by a nominal {\it Airy} pattern was found to mitigate this contamination by over four and two orders of magnitude relative to a dipole and a phased array of dipoles respectively. HERA has significantly based its antenna design principles on these findings in choosing its antenna geometry while paying close attention to the properties of the resulting antenna power pattern. 

In this paper, we investigate the spectral properties of the proposed dish design through their effects on the resultant foreground delay power spectrum and the constraints they place on attenuation required to suppress reflections between the dish-receiver assembly and antenna pairs to within tolerable limits. 

\section{Simulations}\label{sec:sim}

We simulate wide-field visibilities for 19-element HERA from all-sky antenna power pattern and foreground models using the PRISim\footnote{The Precision Radio Interferometry Simulator (PRISim) is publicly available at \url{https://github.com/nithyanandan/PRISim}} software package. The simulations cover 24~hr of observation in {\it drift} mode consisting of 80 accumulations spanning 1080~s each. The total bandwidth is 100~MHz centered on 150~MHz consisting of 256 channels with 390.625~kHz frequency resolution. Models of the antenna power pattern and foregrounds are described below.

\subsection{Antenna Power Pattern}\label{sec:beam-model}

The High Frequency Structural Simulator (HFSS) was used to model the dish and its angular response used in this study. The HFSS model used prime focus optics with a 14~m faceted parabola with a spar f/D ratio of 0.32.  The model has a 1~m central hole in the aluminum surface which is filled with a dielectric material similar to dry soil. The feed used a full PAPER dipole inside of a cylindrical backplane, the backplane is modeled as an aluminum surface. For the metal parts of the dipole, the discs were modeled as aluminum at the actual size, and the arms and terminals were modeled as copper. Dielectric stand-offs and supporting members were included. For the calculations, one pair of arms was excited using a modal port. These models cover a frequency range of 100--200~MHz in intervals of 1~MHz. Refer to DeBoer~et~al.~(2016) for a complete description of the dish model. 

For reference, we use two other models for the antenna power pattern. The first is a nominal {\it Airy} pattern corresponding to a uniformly illuminated circular aperture of 14~m diameter and the second is an achromatic model where the response of the design at 150~MHz of the HFSS model described above was fixed as the hypothetical response at all frequencies covering the entire band. This frequency independent model will be used to isolate the effects of spectral structures in the antenna power pattern (or beam chromaticity) on foreground delay power spectra. Hereafter, we refer to these three beams as `simulated', `{\it Airy}' and `achromatic' models.

In a related series of papers, \citet{neb16} discuss the agreement of these simulated antenna beam patterns with actual measurements and \citet{ewa16} model the reflections and return loss expected in the proposed antenna-receiver assembly. Our focus in this paper is to investigate chromaticity of antenna power patterns from the point of view of their impact on foreground contamination in delay power spectra.

\subsection{Foreground Model}\label{sec:foreground}

Our all-sky foreground model is the same as the one in \citet{thy15a}. It consists of diffuse emission \citep{deo08} and point sources. The latter is obtained from a combination of the NRAO VLA Sky Survey \citep[NVSS;][]{con98} at 1.4~GHz and the Sydney University Molonglo Sky Survey \citep[SUMSS;][]{boc99,mau03} at 843~MHz with a mean spectral index of -0.83. The diffuse sky model has an angular resolution of 13\farcm 74 and a spectral index estimated for every pixel.

\subsection{EoR Model}\label{sec:EoR-model}

For reference, we use two models of EoR. In the first, simulations of the H{\sc i} signal were created using the publically available 21cmFAST\footnote{\url{http://homepage.sns.it/mesinger/DexM\_\_\_21cmFAST.html}} code described in \citet{mes11}. The code uses the excursion set formalism of \citet{fur04a} to generate ionization and 21cm brightness fields for numerous redshifts. The model shown in this paper assumes the same fiducial values of $T_\text{vir}^\text{min} = 2 \times 10^4$~K (virial temperature of minimum mass of dark matter halos that host ionizing sources), $\zeta = 20$ (ionization efficiency), and $R_\text{mfp}=15$~Mpc (mean free path of UV photons) which predicts the redshift of 50\% ionization (and hence a peak in the power spectrum signal) to be at $z=8.5$ as in Ewall-Wice~et~al.(2016; submitted). The second model is borrowed from the simulations described in detail in \citet{lid08}. Hereafter, we refer to these two as EoR models 1 and 2 respectively. 

\section{Chromaticity of Power Pattern}\label{sec:beam-chromaticity}

The equation for delay spectrum describes the mapping between sky location of a foreground object and delay. The chromatic nature (variation with frequency) of the antenna power pattern results in a convolution of the geometrical mapping with the delay response of spectral chromaticity of the power pattern. This can result in a significant spillover of foreground power beyond the horizon delay limits especially in the case of foregrounds near the horizon. 

\subsection{Effect on Foreground Contamination}\label{sec:effects-fgdps}

Since our aim in this paper is to quantify foreground contamination and EoR sensitivity without employing any sophisticated {\it foreground removal} techniques, we turn out attention to {\it foreground avoidance} strategy that employs spectral weighting technique. In such an estimation of power spectrum, specific choices for spectral weighting function, $W(f)$, have been found to be effective in reducing foreground contamination by many orders of magnitude \citep{thy13} and is regularly used in EoR data analysis \citep{par12a}. For instance, a {\it Blackman-Harris} function \citep{har78} has a dynamic range of $\sim$100 -- 120~dB in delay power spectrum and a reduced effective bandwidth, with:
\begin{align}\label{eqn:Beff}
  \epsilon &= \frac{1}{B} \int_{-B/2}^{+B/2} |W(f)|^2\,\dif f, \\
  \textrm{and,}\quad B_\textrm{eff} &= \epsilon\,B,
  % B_\textrm{eff} &= \int_{-B/2}^{+B/2} |W(f)|^2\,\dif f.
\end{align}
where, $1-\epsilon$ is the loss in overall spectral sensitivity and $B_\textrm{eff}$ is the effective bandwidth. For a {\it Blackman-Harris} window function denoted by $W_\textrm{BH}(f)$, $\epsilon \approx 50\%$. Hence, the window spans a total of $2N$ frequency channels each of width $\Delta f$ to have $B_\textrm{eff} \approx N\,\Delta f$.

Considering that the dynamic range required to suppress foreground contamination in the {\it EoR window} may probably be even higher than that provided by a {\it Blackman-Harris} window function, we use a modified version given by the convolution of a {\it Blackman-Harris} window with itself:
\begin{align}\label{eqn:bhw2}
  W(f) &= W_\textrm{BH}(f) \ast W_\textrm{BH}(f).
\end{align}
In its Fourier domain, this window has a response that is obtained by squaring of the {\it Blackman-Harris} window response and thus improves the dynamic range further in the power response by another 100 -- 120~dB while $\epsilon\approx 42\%$. This is an enormous gain in sidelobe characteristics for a negligible loss in over sensitivity relative to a simple {\it Blackman-Harris} weighting. This will ensure that the contamination resulting from spillover of foregrounds along the line-of-sight $k$-modes are limited by intrinsic spectral structures in the foregrounds and the beam patterns and not by sidelobes from spectral weighting. In this paper, we apply this modified spectral weighting function defined in equation~\ref{eqn:bhw2} whenever foregrounds are represented in the Fourier delay domain. 

\begin{figure}[htb]
  \centering
  \includegraphics[width=\linewidth]{window_function_modifications.eps}
  \caption{Choices for spectral weighting functions, $W(f)$ ({\it top}) and their delay power spectrum responses, $|\widetilde{W}(\tau)|^2$ ({\it bottom}). The gray curves correspond to a {\it Blackman-Harris} window while those in black correspond to a {\it Blackman-Harris} window convolved with itself. The overall sensitivity of the former is $\epsilon \approx 50\%$ of a rectangular window while that of the latter is $\epsilon \approx 42\%$. Hence, the latter is narrower than the former. As a result, the main lobe of the convolved window function response is wider than that of a {\it Blackman-Harris} window. However, the sidelobes from the convolved window function are suppressed by more than ten orders of magnitude relative to that of a nominal {\it Blackman-Harris} window for only a slight loss of resolution.}
  \label{fig:window-functions}
\end{figure}

In HERA-19 array layout, there are 30 unique baseline vectors and 8 unique baseline lengths. Fig.~\ref{fig:asm-dps-beam-chromaticity-baselines} shows the delay power spectra of foregrounds on the 8 unique baseline lengths obtained with the aforementioned models for power pattern. In all these panels, the full-band spectrally weighted foreground delay power spectra obtained with achromatic, {\it Airy} and fully chromatic simulated beam patterns are shown in black, red, and blue respectively. The brightening of foreground power near the horizon limits (vertical dotted lines) due to the {\it pitchfork} effect \citep{thy15a,thy15b} is prominently seen in all cases. A clear trend of broadening of spillover-wings outside the horizon limits is seen with increasing chromaticity as the beam is changed from the achromatic to the {\it Airy} to the chromatic simulated model. For instance, the spillover from foreground delay power spectrum obtained with chromatic beam pattern is restricted to $|k_\parallel| \lesssim 0.2\,h$~Mpc$^{-1}$, with the {\it Airy} pattern it is restricted to $|k_\parallel| \lesssim 0.15\,h$~Mpc$^{-1}$, while with the achromatic beam it is restricted to $|k_\parallel| \lesssim 0.12\,h$~Mpc$^{-1}$ even on longest baseline lengths. It is also noted that despite the extreme dynamic range of the employed spectral weighting function, the tail of the foreground spillover at $|k_\parallel| \gtrsim 0.2\,h$~Mpc$^{-1}$ with an {\it Airy} beam pattern is many orders of magnitude higher than that using an achromatic beam while that from the simulated chromatic beam is even higher than from an {\it Airy} pattern by a few orders of magnitude. This signals that the foreground spillover shown are truly limited by intrinsic spectral features in the antenna beam patterns. 

\begin{figure*}[htb]
  \centering
  \includegraphics[width=\linewidth]{asm_foreground_eor_beam_chromaticity_fullband_bhw2.0.eps}
  \caption{Foreground delay power spectra on unique baseline lengths of HERA-19 at arbitrary local sidereal times (LST). The baseline length and orientation (anti-clockwise from East) are annotated at the top right corner of each panel. Black, red and blue curves correspond to delay power spectra obtained with achromatic, {\it Airy} and chromatic simulated antenna beams respectively. The achromatic beam has no spectral structure, the simulated chromatic beam has maximum chromaticity while a nominal {\it Airy} pattern has intermediate level of chromaticity. The {\it pitchfork} effect is clearly visible as peaks around the horizon limits. With increase in chromaticity of the antenna beam the foreground spillover beyond the {\it foreground wedge} becomes progressively worse - the extent of foreground-spillover wings beyond the horizon limits (vertical dotted lines) and the amplitude of spillover beyond $k_\parallel\gtrsim 0.2\,h$~Mpc$^{-1}$ is most severe for the chromatic simulated beam, intermediate for an {\it Airy} pattern and negligible for an achromatic beam.}
  \label{fig:asm-dps-beam-chromaticity-baselines}
\end{figure*}

It is clearly demonstrated that higher the beam chromaticity, the farther the foreground contamination inherently extends along $k_\parallel$. Thus, the chromaticity of antenna beam needs to be carefully controlled in EoR experiments in order to not allow sensitive $k_\parallel$-modes to become inaccessible for EoR H{\sc i} signal detection.

It must be emphasized that such a significant spillover is not limited by the horizon limits of the {\it foreground wedge} because this is caused by spectral structure in the antenna beam pattern in addition to that caused by position-dependent geometric phase in the visibility spectra. As a result delay-based complex deconvolution techniques such as CLEAN \citep{tay99,par09,par12b} that rely on smoothness of foreground spectra and only spectral window shape will not have adequate information to accurately deconvolve intrinsic supra-horizon spillover arising from the chromaticity of the antenna beam. We leave delay power spectrum estimation using {\it foreground removal} strategy that accounts for direction-dependent and beam chromaticity-dependent effects to future work and use only simple {\it foreground avoidance} approach in this study.

\subsection{Constraints on Reflections in the Instrument}\label{sec:constraints-reflectometry}

One of the primary causes for spectral structure in antenna power patterns is reflections in the instrument. Patra~et~al.~2015 (in preparation) and \citet{ewa16} discuss the measured and simulated reflections respectively between a dish and its feed. Reflection between structures and signal paths from different antennas also causes chromaticity in the antenna beam. In this section, we provide cosmologically motivated upper limits on instrument systematics caused by these two types of reflections. % a related discussion by estimating the attenuation in power required to keep the foreground power reflected between adjacent antennas below the expected EoR H{\sc i} signal level for the different models of power patterns used in our study.

The effect of reflections it to shift the measured foreground power to higher modes in $\tau$ (or equivalently in $k_\parallel$) and thus cause further contamination in these higher $k_\parallel$ modes which are considered sensitive for EoR H{\sc i} signal detection. Equivalently, these delay shifts introduce ripples in the spectrum. The net chromaticity in the measurements is the product of spectral structure arising out of the intrinsic nature of foreground emission, the baseline-dependent frequency structure of the fringes, the spectral features in the antenna power pattern besides any other spectral structures in the instrument unaccounted for. Equivalently, in the Fourier domain these factors have a convolving effect. In order to isolate the requirements on beam chromaticity from other sources of chromaticity, we start with an antenna beam in which those spectral structures are absent but the intrinsic chromaticity of foregrounds and the baseline fringes are included. 

% We estimate the attenuation required at specific modes of interest $|k_\parallel|>k_\textrm{p}$ to contain reflected foreground power between antennas below expected EoR H{\sc i} power, both corresponding to a 14.6~m antenna spacing. 

We define the required attenuation on the reflected foreground power as the ratio: 
\begin{align}\label{eqn:attenuation}
  \Gamma_{k_\textrm{p}}(\tau) \geq \min_{|k_\parallel|>k_\textrm{p}}\left\{\frac{P_\textrm{FG}(k_\parallel - \frac{\dif k_\parallel}{\dif \tau}\,\tau)}{P_\textrm{H{\sc i}}(k_\parallel)}\right\}^{1/2},
\end{align}
where, $P_\textrm{H{\sc i}}(k_\parallel)$ is the EoR H{\sc i} power spectrum for the chosen antenna spacing, $P_\textrm{FG}(k_\parallel)$ is the foreground delay power spectrum obtained with the spectral weighting function applied over the full band of the three uniquely oriented 14.6~m antenna spacings further averaged over a 0--12 hr LST range and over both positive and negative $k_\parallel$ modes, $\tau$ is the delay caused by reflections, and $\dif k_\parallel/\dif \tau$ is the {\it jacobian} in the transformation of $\tau$ to $k_\parallel$ (see equation~\ref{eqn:kprll-delay}). Equivalently, $\Gamma_{k_\textrm{p}}(\tau)$ is determined by the requirement that the reflected foreground power obtained by shifting in delay is below the EoR H{\sc i} signal power in line-of-sight spatial scales of interest given by $|k_\parallel|>k_\textrm{p}$. For this analysis constraining reflections, we use the 21cmFAST model at 150~MHz ($z\approx 8.47$) for $P_\textrm{H{\sc i}}(k_\parallel)$.

For reflections caused by dish-feed assembly, we use $P_\textrm{FG}(k_\parallel)$ obtained with the achromatic beam model on a 14.6~m baseline in equation~\ref{eqn:attenuation}. Fig.~\ref{fig:fg-reflections-achrmbeam} shows $\Gamma_{k_\textrm{p}}(\tau)$ (in dB) for $k_\textrm{p}$ chosen to be 0.1~$h$~Mpc$^{-1}$ (solid), 0.15~$h$~Mpc$^{-1}$ (dashed), and 0.2~$h$~Mpc$^{-1}$ (dotted). These curves set an upper limit to the reflected foreground power below the EoR H{\sc i} signal power as a function of delays caused by dish-feed reflections. In other words, it implies that if reflections in the dish-feed assembly are attenuated to levels that lie in the regions below the different shaded regions, such spectral systematics in the instrument will not hinder detection of EoR in those corresponding $k_\parallel$ modes of interest. 

\begin{figure}[htb]
\centering
\includegraphics[width=\linewidth]{spec_on_achrmbeam_foreground_reflected_power_21cmfast_14.6m_150.0_MHz_subband_v2.eps}
\caption{Lower limit on attenuation of reflected foreground power (in dB) from dish-feed reflections required to keep the reflected foreground power below EoR H{\sc i} signal power for all $k_\parallel$-modes greater than $0.1\,h$~Mpc$^{-1}$ (solid), $0.15\,h$~Mpc$^{-1}$ (dashed) and $0.2\,h$~Mpc$^{-1}$ (dotted). This is obtained on 14.6~m antenna spacing for a 21cmFAST EoR model at 150~MHz ($z\approx 8.47$) and foreground power obtained with an achromatic antenna beam model. When the attenuation on dish-feed reflections is higher than these limits (outside the shaded regions), EoR will be detectable in respective $k_\parallel$-modes despite these reflections.}
\label{fig:fg-reflections-achrmbeam}
\end{figure}

The elbow-shaped turnover is a measure of the most severe requirement on attenuation of reflections. This depends sensitively on the chosen $k_\parallel$ modes. For instance, the attenuation required is $\gtrsim 54$~dB at $\sim 200$~ns for $k_\textrm{p}=0.1\,h$~Mpc$^{-1}$, $\gtrsim 56$~dB at $\sim 300$~ns for $k_\textrm{p}=0.15\,h$~Mpc$^{-1}$ and $\gtrsim 58$~dB at $\sim 400$~ns for $k_\textrm{p}=0.2\,h$~Mpc$^{-1}$. Measurements by Patra et al. (2016) for HERA dish-feed assembly show that the return losses lie well within the regions excluded by the shaded regions thus implying that the attenuation is well above the lower limit and hence will not prevent EoR detection.

Similarly, for reflections arising out of structures and interfaces across different antennas, we use $P_\textrm{FG}(k_\parallel)$ obtained with the simulated chromatic and {\it Airy} antenna beams on 14.6~m baselines. These beams do not include spectral features from antenna-to-antenna reflections. Thus $\Gamma_{k_\textrm{p}}(\tau)$ provides lower limit for attenuation of antenna-to-antenna reflections. Fig.~\ref{fig:fg-reflections} is similar to Fig.~\ref{fig:fg-reflections-achrmbeam} with constraints for values of $k_\textrm{p}$ chosen to be 0.1~$h$~Mpc$^{-1}$ (left), 0.15~$h$~Mpc$^{-1}$ (middle), and 0.2~$h$~Mpc$^{-1}$ (right). Lower limits are estimated for {\it Airy} (dashed lines) and simulated chromatic beams (solid lines).

\begin{figure*}[htb]
\centering
\includegraphics[width=\linewidth]{spec_on_foreground_reflected_power_21cmfast_14.6m_150.0_MHz_subband_v2.eps}
\caption{Lower limit on attenuation of reflected foreground power (in dB) from antenna-to-antenna reflections required to keep the reflected foreground power below EoR H{\sc i} signal power for all $k_\parallel$-modes greater than $0.1\,h$~Mpc$^{-1}$ (left), $0.15\,h$~Mpc$^{-1}$ (middle) and $0.2\,h$~Mpc$^{-1}$ (right). This is obtained on 14.6~m antenna spacing for a 21cmFAST EoR model at 150~MHz ($z\approx 8.47$) and foreground power obtained with an {\it Airy} (dashed) and simulated chromatic (solid) antenna beam models. Regions excluding shaded regions indicate EoR will be detectable in respective $k_\parallel$-modes despite these reflections. Increase in beam chromaticity makes the requirement on attenuation more severe. For instance, $k_\parallel > 0.1\,h$~Mpc$^{-1}$ modes will be inaccessible with the simulated chromatic beam if these reflections are taken into account. However, for the same beam the instrument will enjoy a much higher tolerance despite these reflections if the modes of interest are $k_\parallel > 0.2\,h$~Mpc$^{-1}$ for EoR detection. Owing to lower chromaticity, an {\it Airy} beam offers higher tolerance to reflections compared to the simulated chromatic beam.}
\label{fig:fg-reflections}
\end{figure*}

Increase in beam chromaticity makes the lower limits on attenuating reflections more severe relative to that from an achromatic beam. For instance, for $k_\textrm{p}=0.1\,h$~Mpc$^{-1}$ there is no delay at which the required attenuation is unconstrained to the left of the elbow-shaped turnover, including at $\tau=0$. This means that when additional chromaticity from antenna-to-antenna reflections is taken into account, the requirement that all $k_\parallel$-modes with $k_\parallel > 0.1\,h$~Mpc$^{-1}$ be accessible for EoR signal detection will not be satisfied with the currently simulated chromatic beam for HERA. For $k_\textrm{p}=0.15\,h$~Mpc$^{-1}$, there is a narrow range of allowed attenuation which is unconstrained for $\tau \lesssim 50$~ns. However, for $k_\parallel > 0.2\,h$~Mpc$^{-1}$, the foreground power spectrum using simulated chromatic dish beam will have enough room to be effective for EoR signal detection despite additional chromaticity arising from antenna-to-antenna reflections. 

It may be noted here that designing a dish whose antenna beam closely resembles a nominal {\it Airy} pattern will have a significantly higher ceiling of tolerance for allowing antenna-to-antenna reflections and yet remain very effective for EoR signal detection. The HERA collaboration is actively pursuing constant improvement of its dish design to keep minimize limitations from such chromatic systematics.

\section{EoR Sensitivity}\label{sec:eor-sensitivity}

In line with the {\it foreground avoidance} strategy, Fig.~\ref{fig:asm-dps-beam-chromaticity-baselines-150-subband} shows delay spectra after applying a {\it Blackman-Harris} spectral weighting function defined in equation~\ref{eqn:bhw2} with $B_\textrm{eff}=10$~MHz. The 8 panels correspond to those shown in Fig.~\ref{fig:asm-dps-beam-chromaticity-baselines}. As a result of narrowed subband, the resolution in delay space (and $k_\parallel$) has coarsened. The cyan and gray lines denote power spectra of EoR models 1 and 2 respectively. 

\begin{figure*}[htb]
\centering
\subfloat[][Power patterns]{\label{fig:asm-dps-beam-chromaticity-baselines-150-subband}\includegraphics[width=\linewidth]{sim_asm_foreground_eor_beam_chromaticity_150.0_MHz_subband_win_method_bhw2.0.eps}} \\
\subfloat[][Power patterns]{\label{fig:asm-dps-beam-chromaticity-baselines-170-subband}\includegraphics[width=\linewidth]{sim_asm_foreground_eor_beam_chromaticity_170.0_MHz_subband_win_method_bhw2.0.eps}}
\label{fig:subbands}
\end{figure*}

% \begin{figure*}[htb]
% \centering
% \includegraphics[width=\linewidth]{sim_asm_foreground_eor_beam_chromaticity_150.0_MHz_subband_win_method_bhw2.0.eps}
% \caption{Effect of chromaticity of antenna power pattern on foreground delay power spectra obtained from the 150~MHz subband for different baseline lengths.}
% \label{fig:asm-dps-beam-chromaticity-baselines-150-subband}
% \end{figure*}

% Firstly, it is noted that the coarsening of delay resolution significantly absorbs the distinct differences caused by beams of different chromaticities as seen in the full band foreground delay spectra outside the {\it foreground wedge} in Fig.~\ref{fig:asm-dps-beam-chromaticity-baselines}. Secondly, it is also found that barring the shortest baselines ($|\boldsymbol{b}| \leq $~29.2~m), which have the most severe foreground contamination especially from diffuse emission, foreground contamination on baselines longer than $\gtrsim 29.2$~m is lesser than EoR signal power for $|k_\parallel| \gtrsim 0.22\,h$~Mpc$^{-1}$ in either of the models. 

% Fig.~\ref{fig:asm-dps-beam-chromaticity-baselines-170-subband} is similar to Fig.~\ref{fig:asm-dps-beam-chromaticity-baselines-150-subband} except it is obtained with a subband {\it Blackman-Harris} window function centered on 170~MHz. In this subband relative to that centered on 150~MHz, the ratio of powers of the EoR signal and foregrounds is even higher outside the wedge thus indicating a stronger potential for direct detectability without the use of optimal power spectrum estimation techniques on almost all baseline lengths. 

% % \begin{figure*}[htb]
% % \centering
% % \includegraphics[width=\linewidth]{sim_asm_foreground_eor_beam_chromaticity_170.0_MHz_subband_win_method_bhw2.0.eps}
% % \caption{Effect of chromaticity of antenna power pattern on foreground delay power spectra obtained from the 170~MHz subband for different baseline lengths.}
% % \label{fig:asm-dps-beam-chromaticity-baselines-170-subband}
% % \end{figure*}

% It must be emphasized that these are obtained only with windowing techniques and do not represent the best sensitivity achievable with sophisticated tools such as optimal covariance-based weighting schemes \citep{liu14a,liu14b,ali15,dil15}. Thus assuming the data is limited by foregrounds and not by thermal noise, this demonstrates that even with no sophistication in power spectrum estimation, a direct detection based on simple windowing and delay spectrum technique is possible on a majority of HERA baselines and will further allow distinguishing between different EoR models. This finding holds even in the most conservative scenario where the modeled beam pattern with most severe chromaticity inherently results in significant and inherent foreground contamination outside the {\it foreground wedge}.

\begin{figure*}[htb]
\centering
\subfloat[][Power patterns]{\label{fig:21cmfast-fg-ratio-150}\includegraphics[width=0.45\linewidth]{eor_fg_ratio_llim_21cmfast_chrmbeam_sim_150.0_MHz_subband.eps}}
\subfloat[][Power patterns]{\label{fig:lidz-fg-ratio-150}\includegraphics[width=0.45\linewidth]{eor_fg_ratio_llim_lidz_chrmbeam_sim_150.0_MHz_subband.eps}} \\
\subfloat[][Power patterns]{\label{fig:21cmfast-fg-ratio-170}\includegraphics[width=0.45\linewidth]{eor_fg_ratio_llim_21cmfast_chrmbeam_sim_170.0_MHz_subband.eps}}
\subfloat[][Power patterns]{\label{fig:lidz-fg-ratio-170}\includegraphics[width=0.45\linewidth]{eor_fg_ratio_llim_lidz_chrmbeam_sim_170.0_MHz_subband.eps}}
\caption{}
\label{fig:eor-fg-ratios}
\end{figure*}


% Owing to the chromaticity of power patterns, the spillover from foregrounds extends beyond the horizon limits, most severely in the simulated disk pattern. As a result the residuals have significant power remaining on either side of the horizon limits since the window for deconvolution is narrower than the extent of spillover along $k_\parallel$. The two peaks on either side is most severe for the simulated disk pattern followed by that for the {\it Airy} disk pattern and least for an achromatic power pattern. 

% We investigate the effect of chromaticity of power pattern in a foreground avoidance strategy that avoids the aforementioned deconvolution of delay spectrum. Application of window functions with high sidelobe suppression have been discussed for mitigating foreground contamination \citep{thy13}. In line with the foreground avoidance strategy, the visibilities from the full band foreground simulation (whose delay power spectrum is shown on the top left) are multiplied by a Blackman-Harris window function of 10~MHz effective bandwidth. The delay power spectra so produced are shown in the panel on the bottom left. Due to the decreased effective bandwidth the resolution of the delay power spectrum appears decreased. Hence, only the $|k_\parallel| \gtrsim 0.14\,h$~Mpc$^{-1}$ modes remain accessible. Due to the overall lower amplitude of the {\it Airy} power pattern relative to the simulated dish power pattern and the achromatic pattern, the sidelobe levels of foreground delay power spectra from an {\it Airy} pattern are lower than the other models by almost two orders of magnitude. 

% Following a delay-deconvolution based foreground removal strategy, we apply the Blackman-Harris window of effective bandwidth 10~MHz on the unsubtracted foreground residuals (top right panel) in frequency domain and obtain the delay power spectrum shown in the bottom right panel. It is noted that lesser the chromaticity the more effective the deconvolution process is in lowering the residuals in and around the horizon limits. For instance, the peak of the delay power spectrum using the achromatic beam is lowered by four orders of magnitude relative to the original peak whereas the residuals of delay power spectra with chromatic {\it Airy} and simulated power patterns is lowered only by two orders of magnitude. However, owing to the ineffective subtraction of the response extended beyond the horizon limits, the sidelobe levels have not been lowered significantly and are within an order of magnitude of each other.

% In both strategies, the EoR H{\sc i} signal power spectrum obtained by simulations with 21cmFAST (Mesinger et al. 2008) is shown for reference (dashed black lines) in the bottom panels. 

\section{Summary}\label{sec:summary}

\acknowledgments

This work was supported by the U. S. National Science Foundation (NSF) through award AST-1109257. DCJ is supported by an NSF Astronomy and Astrophysics Postdoctoral Fellowship under award AST-1401708. 

% \appendix

% \par\bigskip
\bibliographystyle{apj}
\bibliography{eor}

\end{document}
